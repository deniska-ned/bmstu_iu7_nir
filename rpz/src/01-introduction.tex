\chapter*{ВВЕДЕНИЕ}
\addcontentsline{toc}{chapter}{ВВЕДЕНИЕ}

    В современном мире все большую популярность приобретают распределенные системы. Переход к ним аргументируется лучшей масштабируемостью и отказоустойчивостью. Однако распределенные системы порождают новые задачи, связанные с согласованными принятиями решений между узлами системы. Принятие таких решений описывают алгоритмы достижения консенсуса.
    
    Целью данной работы является обзор существующих алгоритмов консенсуса.
    
    Для достижения поставленной цели требуется решить следующие задачи:
    
    \begin{itemize}
        \item определить основные термины, связанные алгоритмами консенсуса;
        \item рассмотреть существующие алгоритмы;
        \item выделить критерии классификации алгоритмов;
        \item провести классификацию алгоритмов.
    \end{itemize}